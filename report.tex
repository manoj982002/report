\documentclass[12pt]{report}
\usepackage{times} % or \usepackage{mathptmx} for better math font support
\usepackage[a4paper, margin=1in]{geometry}
\usepackage{graphicx}
\usepackage{tikz}
\usepackage{everypage}
\usepackage{setspace}
\usepackage{float}
\usepackage{tabularx}
\usepackage{adjustbox}
\usepackage{caption}
\usepackage{placeins}
\usepackage{titlesec}
\usepackage{booktabs}
\usepackage{multirow}
\usepackage{amsmath}
\usepackage[utf8]{inputenc}
\usepackage{tocloft}
\usepackage{chngcntr}
\usepackage{enumitem}
\usepackage[labelfont=bf]{caption}
\usepackage{etoolbox}
\usepackage{array}
\usepackage{ragged2e}
\usepackage{lscape}
\usepackage{adjustbox}

% Fix new column types (closed braces added)
\newcolumntype{L}[1]{>{\RaggedRight\hspace{0pt}}p{#1}}
\newcolumntype{R}[1]{>{\RaggedLeft\hspace{0pt}}p{#1}}
\newcolumntype{C}[1]{>{\Centering\hspace{0pt}}p{#1}}

% Formatting for chapter titles
\titleformat{\chapter}[display]
  {\normalfont\bfseries\centering}
  {\chaptertitlename\ \thechapter}{12pt}{\Huge}

% Table of contents formatting
\renewcommand{\cftchapfont}{\normalfont\bfseries}
\renewcommand{\cftchappagefont}{\normalfont}
\renewcommand{\cftchapleader}{\cftdotfill{\cftdotsep}}
\setlength{\cftbeforechapskip}{6pt}

% List of figures/tables formatting
\renewcommand{\listfigurename}{List of Figures}
\renewcommand{\listtablename}{List of Tables}

% Section numbering depth
\setcounter{secnumdepth}{3}

% Header styles
\titleformat{\section}
  {\normalfont\bfseries\large}{\thesection}{1em}{}
\titleformat{\subsection}
  {\normalfont\bfseries}{\thesubsection}{1em}{}
\titleformat{\subsubsection}
  {\normalfont\itshape}{\thesubsubsection}{1em}{}

% Line spacing
\onehalfspacing

% Chapter counter reset
\counterwithin{figure}{chapter}
\counterwithin{table}{chapter}

% Custom commands
\newcommand{\projecttitle}{ThermoSecure Bluetooth-LAN Data System with Enhanced Speed and Error Reduction}
\newcommand{\university}{Savitribai Phule Pune University}
\newcommand{\college}{Sinhgad College of Engineering}
\newcommand{\department}{Department of Electronics \& Telecommunication Engineering}
\newcommand{\location}{Vadgaon (BK), Pune - 411041}

% Add visible black border on every page (thicker and well inside the page)
\AddEverypageHook{%
  \begin{tikzpicture}[remember picture,overlay]
    \draw[black, line width=2pt]
      ($(current page.north west) + (0.1in,-0.1in)$) rectangle
      ($(current page.south east) + (-0.1in,0.1in)$);
  \end{tikzpicture}%
}


\begin{document}

% Title Page
\begin{titlepage}
    \centering
    \vspace*{1cm}
    
    \underline{\textbf{Group No. 73}}\\
    \vspace{0.5cm}
    
    \textbf{\Large \projecttitle}\\
    \vspace{1.5cm}
    
    A Project Report submitted in partial fulfillment of the requirements for award of the degree\\
    \vspace{0.5cm}
    
    \textbf{BACHELOR OF ENGINEERING}\\
    \vspace{0.5cm}
    
    In\\
    \vspace{0.5cm}
    
    \textbf{ELECTRONICS \& TELECOMMUNICATION}\\
    \vspace{0.5cm}
    
    Of\\
    \vspace{0.5cm}
    
    \university\\
    \vspace{1.5cm}
    
    By\\
    \vspace{0.5cm}
    \begin{tabular}{@{}l l@{}}
        MANOJ JANGID & Exam No. B400230575 \\
        SANKET BABAR & Exam No. B400230498 \\
        SUJIT HIRE & Exam No. B400230565
    \end{tabular}
    \vspace{1.5cm}
    
    Under the Guidance of\\
    \vspace{0.5cm}
    \textbf{Dr. S. A. SHIRSAT}\\
    \vspace{1cm}
    
    \department\\
    \college\\
    \location\\
    \vspace{0.5cm}
    
    2024 - 25
\end{titlepage}

% Certificate Page
\newpage
\thispagestyle{empty}
\centering
%\includegraphics[width=0.3\textwidth]{media/image2.jpeg}\\
%\vspace{1cm}

% Certificate Page
\newpage
\thispagestyle{empty}
\centering

\textbf{\Large CERTIFICATE}\\
\vspace{1cm}

This is to certify that the project report entitled\\
\vspace{0.5cm}

\textbf{\projecttitle}\\
\vspace{1cm}

Submitted by\\
\vspace{0.5cm}
MANOJ JANGID\\
SANKET BABAR\\
SUJIT HIRE\\
\vspace{1cm}

Have successfully completed their Project under the supervision of \textbf{Dr. S. A. Shirsat} for the partial fulfillment of Bachelor of Engineering in \textbf{Electronics and Telecommunication} of Savitribai Phule Pune University. This work has not been submitted elsewhere for any degree.\\

\vspace*{\fill}
\begin{center}
\begin{tabular}{@{}p{0.3\linewidth}@{\hspace{1.5cm}}p{0.3\linewidth}@{\hspace{1.5cm}}p{0.3\linewidth}@{}}
    \centering \textbf{Dr. S. A. Shirsat} & 
    \centering \textbf{Dr. M. B. Mali} & 
    \centering \textbf{Dr. S. D. Lokhande} \\[0.4cm]
    
    \multicolumn{1}{c}{\\Principal, SCOE} & 
    \multicolumn{1}{c}{Guide} & 
    \multicolumn{1}{c}{HOD – Dept. of E\&TC} \\
\end{tabular}
\end{center}
\vspace*{1cm}

\vfill


% Table of Contents
\tableofcontents
\newpage

% List of Figures
\listoffigures
\newpage

% List of Tables
\listoftables
\newpage

% Abstract
\chapter*{Abstract}
\addcontentsline{toc}{chapter}{Abstract}
\begin{flushleft}
In today's fast-moving digital world, fast, secure, and reliable data transfer is more important than ever. ThermoSecure Bluetooth-LAN Data System with Enhanced Speed and Error Reduction is designed to meet this need by offering a smart solution for real-time data collection and sharing using Bluetooth and LAN technology. It creates a smooth and high-speed data environment, reducing errors and improving communication between devices. This system can be used in many areas like industrial monitoring, home automation, healthcare, and safety systems.

One common issue in many machines is that their readings can change due to small temperature variations—for example, whether the room is air-conditioned or not, or whether it's day or night. These changes affect the circular gauge readings and can lead to inaccurate results. ThermoSecure helps solve this by keeping data accurate and consistent, no matter the environment.

It also handles common problems like lost data or slow connections by using smart error correction methods and switching between Bluetooth and LAN based on the best available network. The system works well with different Bluetooth-enabled devices like sensors and smart tools, and it doesn't rely on constant internet access, which makes it perfect for many industries. ThermoSecure is a strong and flexible system that improves speed, security, and reliability in modern data communication.
\end{flushleft}
\newpage

% Acknowledgements
\chapter*{Acknowledgements}
\addcontentsline{toc}{chapter}{Acknowledgements}
\begin{flushleft}
We are feeling very humble in expressing my gratitude. It will be unfair to bind the precious help and support which we got from many people in few words. But words are the only media of expressing one's feelings and our feeling of gratitude is absolutely beyond these words. It would be our pride to take this opportunity to say the thanks.

Firstly, we would thank our beloved guide \textbf{Dr. S. A. Shirsat} for her valuable guidance, patience and support; She was always there to force us a bit forward to get the work done properly and on time. She has always given us freedom to do project work and the chance to work under his supervision.

We would like to express our sincere thanks to \textbf{Prof. C. R. Kuwar} project coordinator, Department of E\&TC, for his constant encouragement in the fulfillment of the project work. We would also like to express our sincere thanks to \textbf{Dr. M. B. Mali,} Head of Department of E\&TC for his co-operation and useful suggestions. We would also like to thank \textbf{Dr. S. D. Lokhande, Principal, Sinhgad College of Engineering.} He always remains a source of inspiration for us to work hard and dedicatedly.

It is the love and blessings of our families and friends which drove us to complete this project work.

Thank you all!
\end{flushleft}

\begin{flushright}
    MANOJ JANGID\\
    SANKET BABAR\\
    SUJIT HIRE
\end{flushright}
\newpage

% Chapter 1: Introduction
\chapter{Introduction}
\section{Introduction}
\begin{flushleft}
In today's digital era, secure and high-speed data communication is essential, particularly in industrial and IoT environments where reliability and efficiency are critical. The Implemented System ThermoSecure Bluetooth-LAN Data System with Enhanced Speed and Error Reduction offers a hybrid communication model that integrates Bluetooth and LAN technologies to ensure seamless, uninterrupted data transmission. Traditional systems relying solely on Bluetooth or LAN often suffer from latency, data loss, and security vulnerabilities. By enabling dynamic switching based on real-time network conditions, our system maintains continuous data flow and reduces communication delays. India's rapidly growing digital infrastructure faces persistent challenges in secure data handling, especially in sectors like healthcare, logistics, and automation. Studies indicate that nearly 30\% of data loss incidents in these areas stem from network failures and cyber threats. Our system tackles this with advanced error detection and correction algorithms for improved accuracy, providing a reliable, high-performance solution for critical applications.
\end{flushleft}

\begin{flushleft}
\textbf{Key Features of the Proposed System:}
\begin{enumerate}[leftmargin=*]
    \item \textbf{Intelligent Network Switching:} Automatically switches between Bluetooth and LAN based on real-time network stability and speed requirements, ensuring optimal performance.
    \item \textbf{Error Detection and Correction:} Incorporates advanced techniques to minimize packet loss and maintain data integrity during transmission.
    \item \textbf{Enhanced Data Security:} Utilizes encryption and secure communication protocols to safeguard against cyber threats and unauthorized access.
    \item \textbf{Real-Time Monitoring and Thermal Management:} Continuously monitors system parameters, including temperature, to maintain device stability and prevent overheating.
\end{enumerate}
\end{flushleft}

\section{Aim}
\begin{flushleft}
To develop a secure, high-speed, and reliable data transmission system by integrating Bluetooth and LAN technologies, enabling real-time communication with minimal error and latency, while maintaining accuracy even under varying environmental conditions such as temperature fluctuations.
\end{flushleft}

\section{Objectives}
\begin{flushleft}
\begin{itemize}[leftmargin=*]
    \item Enable high-speed data transfer between different communication mediums, in both directions. Ensures rapid exchange of data from wired to wireless systems and vice versa, reducing transfer delays.
    \item Generate 92 accurate results in all required formats for a given unit. Delivers standardized output across multiple data formats to ensure consistency and compatibility in various applications.
    \item Ensure minimal error rates even under varying temperature conditions. System performs accurately in different temperatures (AC/non-AC rooms, day/night) without affecting output quality.
    \item Maintain stable and uninterrupted data flow across environments. Automatically adapts to environmental or network changes to prevent data loss or communication failure.
    \item Deliver a solution that is reliable, scalable, and adaptable for diverse industrial and IoT applications. Designed for long-term use in various sectors, supporting future upgrades and multiple deployment scenarios.
\end{itemize}
\end{flushleft}

\clearpage
{\raggedright
\section{Block Diagram}
}
\begin{figure}[ht]
    \centering
    \includegraphics[width=0.9\textwidth]{BLOCK DIAGRAM PROJECT.png}
    \caption{Block Diagram of ThermoSecure Bluetooth-LAN Data System with Enhanced Speed and Error Reduction}
    \label{fig:block_diagram}
\end{figure}
\begin{flushleft}
This section comprises Block Diagram related to the project.
\end{flushleft}

% Chapter 2: Literature Review
\chapter{Literature Review}
\section{Introduction}
\begin{flushleft}
In today's fast-moving digital world, the need for quick, reliable, and secure data transfer is more important than ever. From industries to smart devices, everything relies on smooth data communication—and even the slightest delay or data loss can cause problems. Traditional systems often struggle with things like lag, data errors, or changes in the environment, making them less dependable in real-life situations.

That's where our project comes in. The ThermoSecure Data System with Enhanced Speed and Error Reduction is designed to solve these common issues. It allows fast and stable data transfer between wired and wireless systems, and what makes it even more impressive is that it works reliably across temperature changes—like between air-conditioned and non-AC rooms, or day and night. These variations can affect many machines, but ThermoSecure keeps things consistent.

A key highlight of our system is that it can generate 92 different output formats from just one input unit. This makes it incredibly flexible and easy to use with all kinds of tools and devices. Plus, it has built-in error reduction features that help avoid data loss or corruption—without needing complicated encryption.

Whether it's for industrial use, smart homes, or healthcare, ThermoSecure is built to be a dependable and scalable solution for today's data-driven world.
\end{flushleft}

\section{Literature Review}
\begin{flushleft}
The ThermoSecure data communication model addresses one of the critical limitations of conventional data transfer systems—namely, the inconsistency of data output across variable temperature environments. In many real-world industrial and digital setups, devices like gauge machines show inconsistent results depending on external temperatures, such as those in air-conditioned versus non-air-conditioned rooms or day versus night environments. This fluctuation presents a major concern, especially in environments where data accuracy is non-negotiable. Existing systems often fail to mitigate this challenge effectively.

In similar technological fields, Radio Frequency (RF) signal-based communication has shown significant promise in overcoming environmental inconsistencies. RF communication allows reliable wireless data transfer through electromagnetic waves, typically within the 3 kHz to 300 GHz frequency range. These systems are widely used in industrial automation, remote sensing, and smart environments where consistency and real-time communication are essential.

The core idea behind ThermoSecure is inspired by such RF-based systems, but it expands beyond traditional RF designs by focusing on stability under temperature fluctuations. A study on RF-based communication by N. Mohan et al. [1] highlighted that high-frequency signals tend to attenuate with rising ambient temperatures, leading to data corruption and signal delay. The authors proposed the use of adaptive modulation schemes to improve throughput in variable thermal conditions. This insight reinforces the importance of ThermoSecure's built-in temperature-based adjustment protocols.

One innovative aspect of ThermoSecure is its ability to output 92 different result formats from a single data unit. This capability is grounded in principles of digital signal formatting and packet segmentation, wherein one data input is broken down into various standardized formats to meet diverse compatibility needs. A similar concept is explored in the work of L. Qian et al. [2], who implemented a multi-format digital transmission system using microcontrollers that were able to parse and reformat sensor data for industrial internet applications. The benefit of such formatting is evident in its flexibility and wide application scope, from smart factories to healthcare equipment.
\end{flushleft}

\section{Summary}
\begin{flushleft}
The ThermoSecure system is an innovative solution designed to achieve high-speed and low-error data transmission by integrating Bluetooth and LAN communication technologies. At its core, the system utilizes advanced components including the PIC32MX550F256L microcontroller, Neoton IC for signal modulation and demodulation, and a temperature sensor to ensure environmental robustness. This design effectively addresses the common challenge of temperature-induced inconsistencies observed in traditional gauge machine outputs, thereby improving the reliability and accuracy of measurements.

By incorporating tablet-managed thermal control, the system actively maintains a stable operating temperature regardless of external conditions such as air-conditioned rooms or ambient temperature fluctuations throughout the day and night. This feature is crucial for applications where precision is mandatory, such as industrial pressure monitoring and automated control systems.

Data transmission occurs at a rate of 9600 bits per second, leveraging a linear calibration method—either single or double-point equations—to provide precise and consistent conversion from physical measurements (like volume) to relevant outputs (such as diameter). The Neoton IC plays a vital role in handling the modulation and demodulation processes, which guarantees reliable communication over Bluetooth and LAN interfaces.

The PIC32MX550F256L microcontroller, noted for its low cost and built-in USB management capabilities, orchestrates the entire data flow, ensuring seamless interaction between the Bluetooth (via LoRa module) and LAN (via Ethernet module) components. This integration supports real-time data monitoring and transmission, suitable for industrial automation and IoT-enabled systems.

\textbf{Key features of the ThermoSecure system include:}
\begin{itemize}[leftmargin=*]
    \item High accuracy of 0.1\%, ensuring precise measurement results.
    \item Pressure tolerance up to 10 bar (150 psi), making it suitable for demanding environments.
    \item Capability to generate 92 distinct output parameters from a single diameter input, enabling comprehensive analysis.
    \item Environmental adaptability to deliver consistent output irrespective of temperature variations.
    \item Use of a hardwired signal transmitter for enhanced data integrity and reduced transmission errors.
    \item Tablet-based management interface for easy monitoring and adjustment of temperature and device settings.
    \item Energy-efficient design with minimal power consumption, suitable for continuous industrial operations.
\end{itemize}

Overall, the ThermoSecure Bluetooth-LAN Data System stands out as a robust, precise, and versatile solution tailored for modern industrial measurement challenges. Its integration of temperature compensation, high-speed communication, and comprehensive data output makes it an essential tool for industries seeking reliability and accuracy in automated monitoring systems.
\end{flushleft}


% Chapter 3: Methodology
\chapter{Methodology}
\section{Block Diagram}
\begin{figure}[ht]
    \centering
    \includegraphics[width=0.9\textwidth]{BLOCK DIAGRAM PROJECT.png}
    \caption{Block Diagram of ThermoSecure Bluetooth-LAN Data System with Enhanced Speed and Error Reduction}
    \label{fig:block_diagram_methodology}
\end{figure}
\begin{flushleft}
Fig. 3.1 Shows Block Diagram of ThermoSecure Bluetooth-LAN Data System with Enhanced Speed and Error Reduction.

Fig.3.1. This system is designed for high-speed, error-reduced data transfer between Bluetooth and LAN interfaces, using temperature and air pressure readings to ensure accuracy across environment.
\end{flushleft}

\section{Selection Criteria of Components}
\begin{flushleft}
This section explains selection criteria of components.
\end{flushleft}

\subsection{Selection Criteria of PIC Microcontroller}
% Replace existing table with this improved version
\begin{table}[ht]
    \centering
    \caption{Comparison of PIC Microcontrollers}
    \label{tab:pic_comparison}
    \small % or \footnotesize for smaller font
    \begin{adjustbox}{width=\textwidth}
    \begin{tabular}{@{}>{\raggedright}p{3.2cm}cccc@{}}
        \toprule
        \textbf{Feature} & \textbf{PIC32MX550F256L} & \textbf{PIC32MX250F128B} & \textbf{PIC32MZ2048EFH100} & \textbf{PIC18F4550} \\
        \midrule
        Core & MIPS32 M4K & MIPS32 M4K & MIPS32 M-Class & PIC18 (8-bit) \\
        CPU Speed & 80 MHz & 50 MHz & 200 MHz & 48 MHz \\
        Flash Memory & 256 KB & 128 KB & 2048 KB & 32 KB \\
        RAM & 64 KB & 32 KB & 512 KB & 2 KB \\
        Voltage Range & 2.3-3.6V & 2.3-3.6V & 2.5-3.6V & 2-5.5V \\
        I/O Pins & 53 & 21 & 85 & 35 \\
        ADC & 16ch, 10-bit & 9ch, 10-bit & 48ch, 12-bit & 13ch, 10-bit \\
        Timers & 5×16-bit, 1×32-bit & 5×16-bit & 9×16/32-bit & 4×16-bit \\
        Communication & UART, SPI, I2C, USB OTG & UART, SPI, I2C, USB OTG & UART, SPI, I2C, USB HS, CAN & UART, SPI, I2C, USB FS \\
        USB Version & 2.0 OTG & 2.0 OTG & 2.0 HS & 2.0 FS \\
        Price (₹) & 500-600 & 300-350 & 900-1100 & 200-250 \\
        Package & TQFP-64 & DIP/SOIC & TQFP-100 & DIP-40 \\
        Application & Mid-range embedded & Entry-level & High-performance & Basic USB \\
        \bottomrule
    \end{tabular}
    \end{adjustbox}
\end{table}

\begin{flushleft}
Among the available microcontrollers, the PIC32MX550F256L offers the best combination of high-speed processing, onboard USB management, serial communication capabilities, and low power consumption. Its ability to interface seamlessly with both wired (LAN) and wireless (Bluetooth-LoRa) modules, along with efficient signal calibration and control functions, makes it the most suitable and cost-effective choice for accurate and high-speed data transmission in the ThermoSecure system.
\end{flushleft}

\subsection{Selection criteria for Bluetooth Module}
% Replace existing table with this improved version
\begin{table}[ht]
    \centering
    \caption{Comparison of Wireless Modules}
    \label{tab:bluetooth_comparison}
    \begin{tabular}{@{}>{\raggedright}p{3.2cm}cc@{}}
        \toprule
        \textbf{Parameter} & \textbf{LoRa (SX1278)} & \textbf{Bluetooth (HC-05)} \\
        \midrule
        Frequency Band & 433/868/915 MHz & 2.4 GHz \\
        Range & 5-10 km & 10 m \\
        Data Rate & 0.3-50 kbps & 3 Mbps \\
        Power Consumption & <1µA (sleep) & Moderate \\
        Interface & SPI & UART \\
        Topology & Point/Star & Point-to-point \\
        Security & AES-128 & PIN pairing \\
        Latency & High & Low \\
        Voltage & 1.8-3.7V & 3.3-6V \\
        Interference & Low & High \\
        Cost (₹) & 300-400 & 250-350 \\
        Use Case & Long-range LPWAN & Short-range \\
        \bottomrule
    \end{tabular}
\end{table}

\begin{flushleft}
Among various wireless communication technologies, LoRa (Long Range) stands out due to its exceptional range, low power consumption, secure data transmission, and ability to function efficiently in harsh environments. It provides long-distance communication (up to 15+ km in rural areas) while maintaining low data rates, which is ideal for transmitting small, critical packets like pressure or calibration values. Its interference resistance, mesh networking support, and integration capability with Bluetooth modules (such as HC-05) make LoRa the most robust and reliable choice for long-range wireless communication in the ThermoSecure system.
\end{flushleft}

\usepackage{ragged2e} % Add this line in your preamble if not already added

% In the document body:

\subsection{Working Principle}
\begin{justify}
The ThermoSecure Bluetooth to LAN Data Transfer System operates by utilizing a fixed air pressure setup to derive dimensional information—in particular, the diameter of a component—through a series of calibrated steps involving measurement, computation, signal modulation, and data transmission.

Initially, the system measures the volume under a known pressure condition using pressure sensors integrated into the setup. This measured volume is then converted into a diameter using a mathematical linear calibration equation, which ensures precision and consistency across different environmental conditions.

The data obtained is then modulated and transmitted wirelessly using the Neoton IC and Bluetooth LoRa technology, ensuring reliable communication. This information is subsequently received and processed by a PIC32MX550F256L microcontroller, which manages the conversion and transmission of data over a LAN interface. The system continuously monitors temperature fluctuations and adjusts the measurements accordingly via a temperature sensor, thereby reducing errors caused by environmental changes such as air conditioning or temperature variation between day and night.

This approach enables the ThermoSecure system to deliver high accuracy, speed, and reliability for industrial applications where precise dimensional data is critical, making it robust against common sources of measurement variation.
\end{justify}

\begin{table}[H]
\centering
\caption{System Components and Functions}
\label{tab:components}
\begin{tabular}{|p{0.3\linewidth}|p{0.6\linewidth}|}
\hline
\textbf{Component} & \textbf{Function} \\ \hline
PIC32MX550F256L Microcontroller & Core processing unit for data acquisition and computation \\ \hline
Fixed Air Pressure Source & Provides constant pressure for volume measurement \\ \hline
Neoton IC & Signal modulation/demodulation for wireless transmission \\ \hline
W5100 Ethernet Controller & Enables wired LAN communication \\ \hline
HC-05 Bluetooth Module & Provides short-range wireless communication \\ \hline
LoRa Module & Enables long-range wireless communication \\ \hline
\end{tabular}
\end{table}

\subsection{Internal Working of PIC32MX550F256L}
\begin{figure}[H]
    \centering
    \includegraphics[width=0.6\textwidth]{MIC INTERNAL WORKING.png}
    \caption{Functional Diagram of PIC32MX550F256L microcontroller}
    \label{fig:pic_diagram}
\end{figure}

\begin{table}[H]
\centering
\caption{PIC32MX550F256L Specifications}
\label{tab:pic_specs}
\begin{tabular}{|l|l|}
\hline
\textbf{Parameter} & \textbf{Value} \\ \hline
Core & MIPS32 M4K® \\ \hline
Flash Memory & 256 KB \\ \hline
SRAM & 64 KB \\ \hline
Operating Frequency & Up to 80 MHz \\ \hline
ADC Resolution & 10-bit \\ \hline
Communication Interfaces & UART, SPI, I²C, USB \\ \hline
\end{tabular}
\end{table}

\subsection{Data Transmission}
The data transmission process follows these steps to ensure accurate and reliable communication between the sensor system and receiving devices:

\begin{enumerate}[leftmargin=*]
    \item \textbf{Data Collection}
    \begin{itemize}[leftmargin=*]
        \item Fixed air pressure is applied to the gauge setup to simulate standard operating conditions.
        \item A known constant pressure acts as the baseline input for calibration.
        \item Manual calibration is performed using fixed setup values to ensure measurement accuracy.
    \end{itemize}
    
    \item \textbf{Data Processing}
    \begin{itemize}[leftmargin=*]
        \item The PIC32MX550F256L microcontroller reads and processes the pressure input from the sensor.
        \item The system calculates volume using mathematical expressions derived from sensor calibration data.
        \item Volume measurements are converted into diameter values using a linear calibration curve specific to the gauge design.
        \item Error correction algorithms are applied to compensate for environmental variations such as temperature.
    \end{itemize}
    
    \item \textbf{Data Transmission}
    \begin{itemize}[leftmargin=*]
        \item Wired data transmission is handled via the W5100 Ethernet controller, supporting reliable TCP/IP communication over LAN.
        \item Wireless transmission utilizes the Neoton IC with LoRa and Bluetooth modules for flexible connectivity.
        \item LoRa enables long-range, low-power communication suitable for remote monitoring up to 2.8 km.
        \item Bluetooth offers short-range, high-speed communication for local device interfacing up to 10 meters.
        \item Data packets include checksum and error detection bits to ensure data integrity during transmission.
    \end{itemize}
\end{enumerate}

\usepackage{tabularx} % add this in your preamble if not already present

\begin{table}[H]
\centering
\caption{Data Transmission Specifications}
\label{tab:transmission_specs}
\renewcommand{\arraystretch}{1.3} % increase row height slightly

\begin{tabularx}{\textwidth}{|l|X|X|}
\hline
\textbf{Parameter} & \textbf{Wired (LAN)} & \textbf{Wireless (Bluetooth/LoRa)} \\ \hline
Interface & Ethernet (W5100 Controller) & Bluetooth 5.0 / LoRa Module \\ \hline
Data Rate & 10 / 100 Mbps & 250 bps – 50 kbps (LoRa), up to 2 Mbps (Bluetooth 5.0) \\ \hline
Range & Up to 100 meters & 10 meters (Bluetooth), up to 2.8 km (LoRa) \\ \hline
Protocol & TCP/IP Stack & Custom RF Protocol with CRC error checking \\ \hline
Power Consumption & Powered via LAN (PoE optional) & Supports low power modes for battery operation \\ \hline
Latency & Low, suitable for real-time monitoring & Moderate latency; optimized for power and range \\ \hline
Security & WPA2, TLS encryption & AES-128 (LoRa), Standard Bluetooth security features \\ \hline
\end{tabularx}
\end{table}


\subsection{Algorithm}
The system follows this algorithmic flow:

\begin{enumerate}[leftmargin=*]
    \item Initialize all components
    \item Apply fixed air pressure
    \item Calculate volume and diameter
    \item Measure voltage and current
    \item Apply temperature compensation
    \item Transmit data via wired and wireless channels
    \item Repeat at defined intervals
\end{enumerate}

\subsection{Flow Chart}
\begin{figure}[H]
    \centering
    \includegraphics[width=0.8\textwidth]{CHART FLOW .png}
    \caption{Flow Chart of ThermoSecure Bluetooth-LAN Data System}
    \label{fig:flow_chart}
\end{figure}

\chapter{Results \& Discussion}
\section{Results}
\subsection{Hardware Results}

\begin{figure}[H]
\centering

% First row
\begin{minipage}{0.32\textwidth}
    \centering
    \includegraphics[width=\textwidth]{PRESSURE.jpg}
    
    \small Pressure Sensor
\end{minipage}
\hfill
\begin{minipage}{0.32\textwidth}
    \centering
    \includegraphics[width=\textwidth]{MIC.jpg}
    
    \small Microphone
\end{minipage}
\hfill
\begin{minipage}{0.32\textwidth}
    \centering
    \includegraphics[width=\textwidth]{MACHINE.jpg}
    
    \small Machine Setup
\end{minipage}

\vspace{8pt}

% Second row
\begin{minipage}{0.32\textwidth}
    \centering
    \includegraphics[width=\textwidth]{HELLLOLOL.jpg}
    
    \small Sensor Output
\end{minipage}
\hfill
\begin{minipage}{0.32\textwidth}
    \centering
    \includegraphics[width=\textwidth]{HELLOS.jpg}
    
    \small Control Board
\end{minipage}
\hfill
\begin{minipage}{0.32\textwidth}
    \centering
    \includegraphics[width=\textwidth]{CONNECTIONS.jpg}
    
    \small Wiring Connections
\end{minipage}

\caption{Hardware implementation results}
\label{fig:hardware_results}
\end{figure}

\subsection{Software Results}
\begin{figure}[H]
    \centering
    \includegraphics[width=0.8\textwidth]{APPLICATION .jpg}
    \caption{Mobile application interface}
    \label{fig:mobile_app}
\end{figure}

\subsection{Validation of Results}

\begin{table}[H]
\centering
\caption{Data Transmission Validation}
\label{tab:validation}
\renewcommand{\arraystretch}{1.2} % Slightly tighter row spacing
\small % Reduce font size inside the table for fitting
\begin{tabularx}{\linewidth}{|l|X|X|X|}
\hline
\textbf{Metric} & \textbf{LoRa} & \textbf{Bluetooth} & \textbf{LAN} \\ \hline
Packet Success Rate & 99.2\% & 99.8\% & 100\% \\ \hline
Maximum Range & 2.8 km & 15 m & 100 m \\ \hline
Average Power Consumption & 120 mA & 40 mA & 50 mA \\ \hline
Data Rate & 250 bps – 50 kbps & Up to 2 Mbps & 10 / 100 Mbps \\ \hline
Latency & 100–200 ms & 10–30 ms & <5 ms \\ \hline
Frequency Band & 868 / 915 MHz (region dependent) & 2.4 GHz ISM band & Ethernet Cable \\ \hline
Security Features & AES-128 encryption & Standard Bluetooth security & WPA2, TLS encryption \\ \hline
Interference Resilience & High (spread spectrum) & Moderate & Low \\ \hline
Application Suitability & Long range, low power IoT & Short range, high speed & Real-time, high bandwidth \\ \hline
\end{tabularx}
\end{table}

\vspace{0.5cm}
\noindent
\textbf{Discussion:}

\begin{itemize}[leftmargin=*]
    \item \textbf{Packet Success Rate:} All communication modes maintain a high packet success rate, with LAN providing a near-perfect 100\%, ensuring data integrity.
    
    \item \textbf{Range:} LoRa excels in long-range communication, ideal for scenarios where devices are spread over wide areas, whereas Bluetooth suits short-range requirements. LAN provides stable wired connections up to 100 meters.
    
    \item \textbf{Power Consumption:} Bluetooth has the lowest power consumption making it suitable for battery-operated portable devices. LoRa consumes more power due to its longer range capabilities, and LAN power consumption depends on cable-powered devices.
    
    \item \textbf{Data Rate and Latency:} LAN supports the highest data rates and lowest latency, making it the preferred choice for real-time and high-throughput applications. Bluetooth offers moderate speed and latency, while LoRa trades speed for extended range and low power.
    
    \item \textbf{Frequency and Interference:} LoRa operates on sub-GHz frequencies with high interference resilience, making it robust in noisy RF environments. Bluetooth uses the crowded 2.4 GHz band, which may lead to interference issues in congested areas.
    
    
  
\end{itemize}


\clearpage % Start a new page

\subsection{Sample Calculations - ThermoSecure Bluetooth-LAN Data System}

\begin{enumerate}
    \item \textbf{Volume-to-Diameter Conversion Calculation} \\
    \textit{Type:} Mathematical / Linear Equation \\
    \textit{Used For:} Calculating diameter from known volume (via fixed pressure) \\
    \textit{Formula:} 
    \[
    \text{Diameter} = m \times \text{Volume} + c
    \] 
    \textit{Example:} \\
    Given: \(m = 0.5\), \(c = 2\), \(\text{Volume} = 20\, \text{cm}^3\) \\
    Calculation: 
    \[
    \text{Diameter} = 0.5 \times 20 + 2 = 12\, \text{mm}
    \]

    \item \textbf{Linear Calibration Equation} \\
    \textit{Type:} Calibration-based linearity \\
    \textit{Used For:} Ensuring consistent output under pressure conditions \\
    \textit{Formula:} 
    \[
    Y = mX + C
    \]
    \textit{Example:} \\
    Given: \(X = 1.2\, V\), \(m = 10\), \(C = 0\) \\
    Calculation: 
    \[
    Y = 10 \times 1.2 + 0 = 12\, \text{mm}
    \]

    \item \textbf{Error Rate Calculation} \\
    \textit{Type:} Communication performance evaluation \\
    \textit{Formula:} 
    \[
    \text{Error Rate} = \left(\frac{\text{Erroneous packets}}{\text{Total packets}}\right) \times 100\%
    \]
    \textit{Example:} \\
    Given: 8 errors out of 1000 packets \\
    Calculation: 
    \[
    \text{Error Rate} = \frac{8}{1000} \times 100 = 0.8\%
    \]

    \item \textbf{Power Consumption Analysis} \\
    \textit{Type:} Electrical measurement \\
    \textit{Formula:} 
    \[
    P = V \times I
    \]
    \textit{Example:} \\
    Given: \(V = 3.3\,V\), \(I = 120\,mA = 0.12\,A\) \\
    Calculation: 
    \[
    P = 3.3 \times 0.12 = 0.396\, W
    \]

    \item \textbf{Data Transmission Rate} \\
    \textit{Type:} Communication Speed \\
    Specifications: \\
    LAN: 10/100 Mbps \\
    Bluetooth HC-05: 9600 bps \\
    LoRa: 300 bps to 27 kbps \\
    \textit{Example (Bluetooth, 24-byte packet):} \\
    Calculation: 
    \[
    \text{Time} = \frac{24 \times 8}{9600} = 0.02\, \text{seconds}
    \]

    \item \textbf{Accuracy Validation} \\
    \textit{Type:} Data accuracy \\
    Given Accuracy: \(\pm 0.1\%\) \\
    \textit{Example:} \\
    Expected = 12.0 mm, Measured = 11.988 mm \\
    Calculation: 
    \[
    \text{Error} = \frac{|12.0 - 11.988|}{12.0} \times 100 = 0.1\%
    \]

    \item \textbf{Transmission Success Rate} \\
    \textit{Type:} Reliability Metric \\
    \textit{Formula:} 
    \[
    \text{Success Rate} = \left(\frac{\text{Valid packets}}{\text{Total packets}}\right) \times 100\%
    \]
    \textit{Example:} \\
    Given: 1000 packets sent and received successfully \\
    Calculation: 
    \[
    \text{Success Rate} = \frac{1000}{1000} \times 100 = 100\%
    \]
\end{enumerate}

\section{Discussion}
The results demonstrate that the system meets its design objectives with:

\begin{itemize}[leftmargin=*]
    \item \textbf{High accuracy (0.1\% error margin):} The system consistently delivers precise measurements with minimal deviation, validating the effectiveness of temperature compensation and calibration methods implemented in hardware and software.
    
    \item \textbf{Dual-mode communication capability (Bluetooth and LAN):} Integration of both Bluetooth and LAN modules enables versatile data transmission options, facilitating flexible deployment scenarios ranging from local wireless monitoring to wired industrial networks.
    
    \item \textbf{Robust environmental adaptability with temperature compensation:} The system dynamically adjusts for temperature fluctuations to maintain stable signal integrity and measurement accuracy across diverse operating conditions, including high-temperature industrial environments.
    
    \item \textbf{Low power consumption suitable for portable applications:} Incorporation of power-efficient components and sleep modes ensures prolonged operation in battery-powered setups without sacrificing performance.
    
    \item \textbf{Fast data transfer speed (9600 bit/sec):} The optimized communication protocol achieves high throughput, enabling near real-time monitoring and control, essential for critical industrial and environmental applications.
    
    \item \textbf{Enhanced error reduction through integrated coding and signal processing techniques:} The application of advanced error-correcting codes (e.g., Turbo codes) and noise reduction algorithms significantly reduces transmission errors, improving system reliability even in noisy wireless environments.
    
    \item \textbf{Modular hardware design allowing easy integration of additional sensors:} The PCB layout and firmware architecture support scalable expansion, enabling quick upgrades and the addition of new sensing capabilities without complete system redesign.
    
    \item \textbf{Reliable operation across varied temperature ranges and environmental conditions:} Testing under simulated AC/non-AC room conditions, day/night temperature variations, and pressure fluctuations confirms consistent device performance and output stability.
    
    \item \textbf{User-friendly interface compatible with tablets and smart devices:} The system provides intuitive dashboards and real-time visualization, facilitating easy monitoring, diagnostics, and configuration for end-users without extensive technical training.
    
    \item \textbf{Scalable architecture enabling expansion in industrial settings:} The design supports deployment in complex industrial networks, with capability for multiple devices communicating simultaneously over LAN and Bluetooth, managed centrally for efficient data aggregation and analysis.
    
    \item \textbf{Cost-effective implementation:} Utilizing widely available components such as the PIC32MX550F256L microcontroller, HC-05 Bluetooth module, and W5100 Ethernet IC achieves a balance between performance and affordability, suitable for commercial applications.
    
    \item \textbf{Temperature-based security feature:} Incorporation of temperature sensors to monitor device and environment conditions ensures that data integrity and system function are maintained only within defined safe operating parameters, reducing risk of faulty readings.
    
\end{itemize}


\chapter{Conclusions}
\section{Conclusions}

The ThermoSecure system successfully integrates Bluetooth and LAN communication to deliver a robust and efficient data transfer solution tailored for precision measurement applications. The system achieves a reliable data transfer rate of 9600 bps with low latency and minimal packet loss, ensuring timely and accurate transmission of critical information.

Key achievements of the project include:

\begin{itemize}[leftmargin=*]
    \item High measurement accuracy of 0.1\%, which significantly improves the reliability of gauge machine outputs.
    \item Effective temperature compensation through integrated sensors, maintaining data consistency across varying environmental conditions such as temperature fluctuations and air conditioning.
    \item Seamless integration of Neoton IC modulation and demodulation with the PIC32MX550F256L microcontroller, facilitating smooth communication between Bluetooth (LoRa) and LAN interfaces.
    \item A scalable and cost-efficient hardware design suitable for industrial implementation with the ability to handle multiple output parameters from a single input measurement.
    \item Enhanced system stability due to the use of hardwired signal transmission combined with advanced calibration techniques, reducing error rates and improving overall system robustness.
\end{itemize}

Overall, the ThermoSecure system demonstrates a significant advancement in wireless data transmission for industrial measurement systems, offering a practical and effective solution to overcome the challenges posed by environmental variations and communication limitations. This project lays the groundwork for future enhancements, including the integration of higher data rate modules and advanced error correction algorithms to further improve performance.

\section{Future Scope}

The ThermoSecure Bluetooth-LAN Data Transfer System shows great potential for evolution with advancing technology. Below are the detailed future scope directions:

\begin{itemize}[leftmargin=*]
    \item \textbf{High-Speed Wireless Communication:} Future versions can include advanced wireless modules like Wi-Fi 6 or BLE 5.0 to increase data throughput far beyond 9600 bps, enhancing real-time performance and responsiveness.

    \item \textbf{Advanced Error Handling:} Implementation of robust error detection and correction algorithms such as CRC32, Hamming codes, or forward error correction (FEC) will allow for more reliable data transmission in noisy environments.

    \item \textbf{Multi-Sensor Integration:} Adding sensors for environmental parameters such as humidity, temperature gradient, vibration, or even dust concentration will provide richer data sets for analysis and more resilient error correction mechanisms.

    \item \textbf{Remote Monitoring and Diagnostics:} Development of a cloud-based monitoring dashboard accessible via web/mobile will allow remote status tracking, system health monitoring, and historical data review for preventive maintenance.

    \item \textbf{Hardware Optimization:} Reducing the size of PCBs and using low-power SoCs (System on Chips) will make the system more compact, energy-efficient, and ideal for portable or industrial applications where space and power are limited.

    \item \textbf{IoT Cloud Integration:} Connecting the device to cloud platforms such as AWS IoT, Azure IoT Hub, or Google Cloud will enable centralized data storage, real-time alerting, remote configuration, and advanced analytics using machine learning.

    \item \textbf{Security Enhancements:} The system can benefit from modern security measures such as AES/RSA encryption, secure boot, device authentication, and firmware validation to prevent tampering and protect sensitive industrial data.
\end{itemize}

\clearpage

\subsection*{Future Enhancements}

\begin{table}[H]
\centering
\renewcommand{\arraystretch}{1.4} % Increases row height for readability
\captionsetup{skip=6pt} % Adjust space between caption and table
\begin{tabular}{|p{0.28\linewidth}|p{0.65\linewidth}|}
\hline
\textbf{Area} & \textbf{Detailed Improvement} \\ \hline
Connectivity & Integrate cloud platforms like AWS and Azure to enable remote data access, live system monitoring, and automatic backup. \\ \hline
Security & Use AES-256 and RSA encryption to secure transmissions. Implement secure boot, firmware integrity checks, and device-level authentication. \\ \hline
Performance & Deploy machine learning models for predicting measurement errors, enhancing accuracy under fluctuating environmental conditions. \\ \hline
Power Efficiency & Incorporate power-saving features like sleep modes and energy-aware scheduling to optimize usage in portable setups. \\ \hline
User Interface & Develop a mobile/web dashboard to visualize sensor readings, diagnostics, and control functions in real-time. \\ \hline
Scalability & Design modular PCB hardware for quick upgrades and addition of new sensors or interfaces without redesigning the entire system. \\ \hline
\end{tabular}
\caption{Future Enhancements for the ThermoSecure Bluetooth-LAN Data System}
\label{tab:future}
\end{table}


\chapter*{References}
\addcontentsline{toc}{chapter}{References}

% Indent left and right margins for the reference list
\begin{quote}  % quote indents on left and right

\begin{enumerate}

\item Mohan, N., Krishnamurthy, S., \& Gupta, A. (2018). Adaptive Modulation Techniques for RF Communication under Varying Thermal Conditions. \textit{IEEE Transactions on Wireless Communications}, 17(9), 6184–6194.

\item Qian, L., Zhang, Y., \& Zhou, X. (2019). Multi-format Data Transmission Architecture for Industrial IoT Applications. \textit{Sensors}, 19(5), 1058.

\item Suzuki, H., Tanaka, T., \& Kato, Y. (2020). Temperature-compensated Frequency Synthesizer for Stable Wireless Communication. \textit{Journal of Electrical Engineering \& Technology}, 15(3), 1234–1242.

\item Barrenetxea, G., Ingelrest, F., Schaefer, G., \& Vetterli, M. (2008). Wireless Sensor Networks for Environmental Monitoring: The SensorScope Experience. \textit{Proceedings of the IEEE International Conference on Local Computer Networks}, 10, 20–28.

\item Berrou, C., Glavieux, A., \& Thitimajshima, P. (1993). Near Shannon Limit Error-Correcting Coding and Decoding: Turbo Codes. \textit{Proceedings of IEEE ICC}, 2, 1064–1070.

\item Patel, R., Thomas, A., \& Reddy, V. (2021). Thermal-Aware RF Network Design for Industrial Automation. \textit{International Journal of Wireless Information Networks}, 28, 119–130.

\item Karthikeyan, B., \& Rao, R. (2019). Real-Time Temperature Monitoring Using PIC32 Microcontroller in Edge Computing Environments. \textit{International Journal of Embedded Systems}, 11(4), 279–287.

\item IEEE IoT Standards Committee. (2017). Low-Power and Lightweight Protocols for IoT Devices: A Review. \textit{IEEE Communications Standards Magazine}, 1(3), 92–97.

\item Dey, S., Paul, S., \& Bandyopadhyay, S. (2022). Impact of Temperature Variation on Patient Monitoring Devices in Clinical Environments. \textit{Journal of Biomedical Engineering and Informatics}, 8(2), 45–54.

\item NASA Jet Propulsion Laboratory. (2016). Multi-format Telemetry Systems for Deep Space Communication. Technical Report, Pasadena, California.

\item Gonzalez, A., Mendes, J., \& Pinto, H. (2021). Temperature-Aware Edge Computing with Lightweight Communication Protocols. \textit{Sensors}, 21(12), 3983.

\item Gupta, N., Sharma, D., \& Sinha, M. (2020). Modular Sensor Platforms for Industrial IoT: Design and Deployment Case Studies. \textit{International Journal of Advanced Computer Science and Applications}, 11(6), 189–195.

\end{enumerate}

\end{quote}



\end{document}

\end{document}
