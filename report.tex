\documentclass[12pt]{report}
\usepackage[a4paper, margin=1in]{geometry}
\usepackage{graphicx}
\usepackage{setspace}
\usepackage{titlesec}
\usepackage{tabularx}
\usepackage{booktabs}
\usepackage{multirow}
\usepackage{amsmath}
\usepackage{lipsum}
\usepackage[utf8]{inputenc}
\usepackage{tocloft}
\usepackage{chngcntr}
\usepackage{enumitem}
\usepackage[labelfont=bf]{caption}
\usepackage{etoolbox}

% Formatting for chapter titles
\titleformat{\chapter}[display]
{\normalfont\bfseries\centering}
{\chaptertitlename\ \thechapter}{12pt}{\Huge}

% Table of contents formatting
\renewcommand{\cftchapfont}{\normalfont\bfseries}
\renewcommand{\cftchappagefont}{\normalfont}
\renewcommand{\cftchapleader}{\cftdotfill{\cftdotsep}}
\setlength{\cftbeforechapskip}{6pt}

% List of figures/tables formatting
\renewcommand{\listfigurename}{List of Figures}
\renewcommand{\listtablename}{List of Tables}

% Section numbering depth
\setcounter{secnumdepth}{3}

% Header styles
\titleformat{\section}
{\normalfont\bfseries\large}{\thesection}{1em}{}
\titleformat{\subsection}
{\normalfont\bfseries}{\thesubsection}{1em}{}
\titleformat{\subsubsection}
{\normalfont\itshape}{\thesubsubsection}{1em}{}

% Line spacing
\onehalfspacing

% Chapter counter reset
\counterwithin{figure}{chapter}
\counterwithin{table}{chapter}

% Custom commands
\newcommand{\projecttitle}{ThermoSecure Bluetooth-LAN Data System with Enhanced Speed and Error Reduction}
\newcommand{\university}{Savitribai Phule Pune University}
\newcommand{\college}{Sinhgad College of Engineering}
\newcommand{\department}{Department of Electronics \& Telecommunication Engineering}
\newcommand{\location}{Vadgaon (BK), Pune - 411041}

\begin{document}

% Title Page
\begin{titlepage}
    \centering
    \vspace*{1cm}
    
    \underline{\textbf{Group No. 73}}\\
    \vspace{0.5cm}
    
    \textbf{\Large \projecttitle}\\
    \vspace{1.5cm}
    
    A Project Report submitted in partial fulfillment of the requirements for award of the degree\\
    \vspace{0.5cm}
    
    \textbf{BACHELOR OF ENGINEERING}\\
    \vspace{0.5cm}
    
    In\\
    \vspace{0.5cm}
    
    \textbf{ELECTRONICS \& TELECOMMUNICATION}\\
    \vspace{0.5cm}
    
    Of\\
    \vspace{0.5cm}
    
    \university\\
    \vspace{1.5cm}
    
    By\\
    \vspace{0.5cm}
    \begin{tabular}{l l}
        MANOJ JANGID & Exam No. B1902303097 \\
        SANKET BABAR & Exam No. B1902303020 \\
        SUJIT HIRE & Exam No. B1902303086 \\
    \end{tabular}
    \vspace{1.5cm}
    
    Under the Guidance of\\
    \vspace{0.5cm}
    \textbf{Dr. S. A. SHIRSAT}\\
    \vspace{1cm}
    
    \department\\
    \college\\
    \location\\
    \vspace{0.5cm}
    
    2024 - 25
\end{titlepage}

% Certificate Page
\newpage
\thispagestyle{empty}
\centering
%\includegraphics[width=0.3\textwidth]{media/image2.jpeg}\\
%\vspace{1cm}

\textbf{\Large CERTIFICATE}\\
\vspace{1cm}

This is to certify that the project report entitled\\
\vspace{0.5cm}

\textbf{\projecttitle}\\
\vspace{1cm}

Submitted by\\
\vspace{0.5cm}
MANOJ JANGID\\
SANKET BABAR\\
SUJIT HIRE\\
\vspace{1cm}

Have successfully completed their Project under the supervision of \textbf{Dr. S. A. Shirsat} for the partial fulfillment of Bachelor of Engineering in \textbf{Electronics and Telecommunication} of Savitribai Phule Pune University. This work has not been submitted elsewhere for any degree.\\
\vspace{1.5cm}

\begin{tabular}{l l}
    \textbf{Dr. S. A. Shirsat} & \textbf{Dr. M. B. Mali} \\
    Guide & HOD - Dept. of E\&TC \\[1cm]
    & \textbf{Dr. S. D. Lokhande} \\
    & Principal \\
\end{tabular}
\vfill

% Approval signatures would go here



% Table of Contents
\tableofcontents
\newpage

% List of Figures
\listoffigures
\newpage

% List of Tables
\listoftables
\newpage

% Abstract
\chapter*{Abstract}
\addcontentsline{toc}{chapter}{Abstract}
In today's fast-moving digital world, fast, secure, and reliable data transfer is more important than ever. ThermoSecure Bluetooth-LAN Data System with Enhanced Speed and Error Reduction is designed to meet this need by offering a smart solution for real-time data collection and sharing using Bluetooth and LAN technology. It creates a smooth and high-speed data environment, reducing errors and improving communication between devices. This system can be used in many areas like industrial monitoring, home automation, healthcare, and safety systems.

One common issue in many machines is that their readings can change due to small temperature variations—for example, whether the room is air-conditioned or not, or whether it's day or night. These changes affect the circular gauge readings and can lead to inaccurate results. ThermoSecure helps solve this by keeping data accurate and consistent, no matter the environment.

It also handles common problems like lost data or slow connections by using smart error correction methods and switching between Bluetooth and LAN based on the best available network. The system works well with different Bluetooth-enabled devices like sensors and smart tools, and it doesn't rely on constant internet access, which makes it perfect for many industries. ThermoSecure is a strong and flexible system that improves speed, security, and reliability in modern data communication.
\newpage

% Acknowledgements
\chapter*{Acknowledgements}
\addcontentsline{toc}{chapter}{Acknowledgements}
We are feeling very humble in expressing my gratitude. It will be unfair to bind the precious help and support which we got from many people in few words. But words are the only media of expressing one's feelings and our feeling of gratitude is absolutely beyond these words. It would be our pride to take this opportunity to say the thanks.

Firstly, we would thank our beloved guide \textbf{Dr. S. A. Shirsat} for her valuable guidance, patience and support; She was always there to force us a bit forward to get the work done properly and on time. She has always given us freedom to do project work and the chance to work under his supervision.

We would like to express our sincere thanks to \textbf{Prof. C. R. Kuwar} project coordinator, Department of E\&TC, for his constant encouragement in the fulfillment of the project work. We would also like to express our sincere thanks to \textbf{Dr. M. B. Mali,} Head of Department of E\&TC for his co-operation and useful suggestions. We would also like to thank \textbf{Dr. S. D. Lokhande, Principal, Sinhgad College of Engineering.} He always remains a source of inspiration for us to work hard and dedicatedly.

It is the love and blessings of our families and friends which drove us to complete this project work.

Thank you all!

\begin{flushright}
    MANOJ JANGID\\
    SANKET BABAR\\
    SUJIT HIRE
\end{flushright}
\newpage

% Chapter 1: Introduction
\chapter{Introduction}
\section{Introduction}
This system provides a non-invasive solution for continuous glucose monitoring, reducing the discomfort of traditional methods. The MAX30100 sensor measures oxygen saturation and heart rate, which the ESP32 microcontroller processes to estimate glucose levels. Readings are displayed on an LCD screen and transmitted via Bluetooth (HC-06) to a mobile app. Developed with FlutterFlow and Android Studio, the app offers real-time data visualization and remote monitoring. Patient data is securely stored in Firebase for long-term tracking. A DS18B20 temperature sensor enhances accuracy by providing additional physiological context. To ensure reliability, the system is validated against the AccuCheck invasive device, minimizing deviations. This approach offers a user-friendly, comfortable, and effective alternative for diabetes management.

\section{Aim}
To develop a secure, high-speed, and reliable data transmission system by integrating Bluetooth and LAN technologies, enabling real-time communication with minimal error and latency, while maintaining accuracy even under varying environmental conditions such as temperature fluctuations.

\section{Objectives}
\begin{itemize}
    \item Enable high-speed data transfer between different communication mediums, in both directions. Ensures rapid exchange of data from wired to wireless systems and vice versa, reducing transfer delays.
    \item Generate 92 accurate results in all required formats for a given unit. Delivers standardized output across multiple data formats to ensure consistency and compatibility in various applications.
    \item Ensure minimal error rates even under varying temperature conditions. System performs accurately in different temperatures (AC/non-AC rooms, day/night) without affecting output quality.
    \item Maintain stable and uninterrupted data flow across environments. Automatically adapts to environmental or network changes to prevent data loss or communication failure.
    \item Deliver a solution that is reliable, scalable, and adaptable for diverse industrial and IoT applications. Designed for long-term use in various sectors, supporting future upgrades and multiple deployment scenarios.
\end{itemize}

\section{Block Diagram}
\begin{figure}[ht]
    \centering
    \includegraphics[width=0.9\textwidth]{media/image3.png}
    \caption{Block Diagram of ThermoSecure Bluetooth-LAN Data System with Enhanced Speed and Error Reduction}
    \label{fig:block_diagram}
\end{figure}
This section comprises Block Diagram related to the project

% Chapter 2: Literature Review
\chapter{Literature Review}
\section{Introduction}
In today's fast-moving digital world, the need for quick, reliable, and secure data transfer is more important than ever. From industries to smart devices, everything relies on smooth data communication—and even the slightest delay or data loss can cause problems. Traditional systems often struggle with things like lag, data errors, or changes in the environment, making them less dependable in real-life situations.

That's where our project comes in. The ThermoSecure Data System with Enhanced Speed and Error Reduction is designed to solve these common issues. It allows fast and stable data transfer between wired and wireless systems, and what makes it even more impressive is that it works reliably across temperature changes—like between air-conditioned and non-AC rooms, or day and night. These variations can affect many machines, but ThermoSecure keeps things consistent.

A key highlight of our system is that it can generate 92 different output formats from just one input unit. This makes it incredibly flexible and easy to use with all kinds of tools and devices. Plus, it has built-in error reduction features that help avoid data loss or corruption—without needing complicated encryption.

Whether it's for industrial use, smart homes, or healthcare, ThermoSecure is built to be a dependable and scalable solution for today's data-driven world.

\section{Literature Review}
The ThermoSecure data communication model addresses one of the critical limitations of conventional data transfer systems—namely, the inconsistency of data output across variable temperature environments. In many real-world industrial and digital setups, devices like gauge machines show inconsistent results depending on external temperatures, such as those in air-conditioned versus non-air-conditioned rooms or day versus night environments. This fluctuation presents a major concern, especially in environments where data accuracy is non-negotiable. Existing systems often fail to mitigate this challenge effectively.

In similar technological fields, Radio Frequency (RF) signal-based communication has shown significant promise in overcoming environmental inconsistencies. RF communication allows reliable wireless data transfer through electromagnetic waves, typically within the 3 kHz to 300 GHz frequency range. These systems are widely used in industrial automation, remote sensing, and smart environments where consistency and real-time communication are essential.

The core idea behind ThermoSecure is inspired by such RF-based systems, but it expands beyond traditional RF designs by focusing on stability under temperature fluctuations. A study on RF-based communication by N. Mohan et al. [1] highlighted that high-frequency signals tend to attenuate with rising ambient temperatures, leading to data corruption and signal delay. The authors proposed the use of adaptive modulation schemes to improve throughput in variable thermal conditions. This insight reinforces the importance of ThermoSecure's built-in temperature-based adjustment protocols.

One innovative aspect of ThermoSecure is its ability to output 92 different result formats from a single data unit. This capability is grounded in principles of digital signal formatting and packet segmentation, wherein one data input is broken down into various standardized formats to meet diverse compatibility needs. A similar concept is explored in the work of L. Qian et al. [2], who implemented a multi-format digital transmission system using microcontrollers that were able to parse and reformat sensor data for industrial internet applications. The benefit of such formatting is evident in its flexibility and wide application scope, from smart factories to healthcare equipment.

% Continue the literature review content here...

\section{Summary}
The ThermoSecure system is designed to enable high-speed and low-error data transmission from Bluetooth to LAN by integrating Neoton IC programming, a PIC32MX550F256L microcontroller, and hardwired signal transmission. The system addresses temperature-based inconsistencies often seen in traditional gauge machine outputs by incorporating a temperature sensor and tablet-managed thermal control to maintain stable readings regardless of the environment (e.g., AC vs. non-AC, day vs. night).

The data is transmitted at 9600 bit/sec using a linear calibration method (single or double point equation) to ensure accuracy. The Neoton IC handles both modulation and demodulation for reliable communication. The PIC microcontroller, with built-in USB management and low-cost efficiency, controls the flow of data between Bluetooth (LoRa) and LAN (Ethernet) modules.

Key features include:
\begin{itemize}
    \item 0.1\% accuracy
    \item 10 bar (150 psi) pressure tolerance
    \item 92 output parameters for one diameter input
    \item Perfect output in any environment
\end{itemize}

% Chapter 3: Methodology
\chapter{Methodology}
\section{Block Diagram}
\begin{figure}[ht]
    \centering
    \includegraphics[width=0.9\textwidth]{media/image3.png}
    \caption{Block Diagram of ThermoSecure Bluetooth-LAN Data System with Enhanced Speed and Error Reduction}
    \label{fig:block_diagram_methodology}
\end{figure}
Fig. 3.1 Shows Block Diagram of ThermoSecure Bluetooth-LAN Data System with Enhanced Speed and Error Reduction.

Fig.3.1. This system is designed for high-speed, error-reduced data transfer between Bluetooth and LAN interfaces, using temperature and air pressure readings to ensure accuracy across environment.

\section{Selection Criteria of Components}
This section explains selection criteria of components

\subsection{Selection Criteria of PIC Microcontroller}
\begin{table}[ht]
    \centering
    \caption{Comparison of PIC Microcontrollers}
    \label{tab:pic_comparison}
    \begin{tabular}{lcccc}
        \toprule
        Feature / Parameter & PIC32MX550F256L & PIC32MX250F128B & PIC32MZ2048EFH100 & PIC18F4550 \\
        \midrule
        Core & MIPS32 M4K & MIPS32 M4K & MIPS32 M-Class & PIC18 (8-bit) \\
        CPU Speed & 80 MHz & 50 MHz & 200 MHz & 48 MHz \\
        Program Memory (Flash) & 256 KB & 128 KB & 2048 KB & 32 KB \\
        RAM & 64 KB & 32 KB & 512 KB & 2 KB \\
        Operating Voltage & 2.3V-3.6V & 2.3V-3.6V & 2.5V-3.6V & 2V-5.5V \\
        I/O Pins & 53 & 21 & Up to 85 & 35 \\
        ADC & 16 ch, 10-bit & 9 ch, 10-bit & 48 ch, 12-bit & 13 ch, 10-bit \\
        Timers & 5x16-bit, 1x32-bit & 5x16-bit & 9x16/32-bit & 4x16-bit \\
        Communication & UART, SPI, I2C, USB OTG & UART, SPI, I2C, USB OTG & UART, SPI, I2C, USB HS, CAN & UART, SPI, I2C, USB FS \\
        USB & USB 2.0 OTG & USB 2.0 OTG & USB 2.0 High Speed & USB 2.0 Full Speed \\
        Price (approx) & ₹500-600 & ₹300-350 & ₹900-1100 & ₹200-250 \\
        Package & TQFP-64 & DIP/SOIC & TQFP-100 & DIP-40 \\
        Target Application & Mid-range embedded & Entry-level embedded & High-performance & Basic USB devices \\
        \bottomrule
    \end{tabular}
\end{table}

Among the available microcontrollers, the PIC32MX550F256L offers the best combination of high-speed processing, onboard USB management, serial communication capabilities, and low power consumption. Its ability to interface seamlessly with both wired (LAN) and wireless (Bluetooth-LoRa) modules, along with efficient signal calibration and control functions, makes it the most suitable and cost-effective choice for accurate and high-speed data transmission in the ThermoSecure system.

\subsection{Selection criteria for Bluetooth Module}
\begin{table}[ht]
    \centering
    \caption{Comparison of Bluetooth Modules}
    \label{tab:bluetooth_comparison}
    \begin{tabular}{lcc}
        \toprule
        Parameter & LoRa (RA-02 SX1278) & Bluetooth (HC-05) \\
        \midrule
        Frequency Band & 433/868/915 MHz & 2.4 GHz \\
        Communication Range & 5-10 km (open) & ~10 meters \\
        Data Rate & 0.3-50 kbps & Up to 3 Mbps \\
        Power Consumption & Very Low (<1µA sleep) & Moderate \\
        Interface & SPI & UART \\
        Topology & Point-to-point, star & Point-to-point \\
        Encryption Support & AES-128 & Basic PIN pairing \\
        Latency & Higher & Low \\
        Operating Voltage & 1.8V-3.7V & 3.3V-6V \\
        Interference & Low (sub-GHz) & High (2.4 GHz) \\
        Cost (Approx.) & ₹300-400 & ₹250-350 \\
        Use Case Suitability & Long-range, low power & Short-range, high-speed \\
        \bottomrule
    \end{tabular}
\end{table}

Among various wireless communication technologies, LoRa (Long Range) stands out due to its exceptional range, low power consumption, secure data transmission, and ability to function efficiently in harsh environments. It provides long-distance communication (up to 15+ km in rural areas) while maintaining low data rates, which is ideal for transmitting small, critical packets like pressure or calibration values. Its interference resistance, mesh networking support, and integration capability with Bluetooth modules (such as HC-05) make LoRa the most robust and reliable choice for long-range wireless communication in the ThermoSecure system.

\section{Methodology/Working}
\subsection{Working Principle}
The ThermoSecure Bluetooth to LAN Data Transfer System operates by utilizing a fixed air pressure setup to derive dimensional information—in particular, the diameter of a component—through a series of calibrated steps involving measurement, computation, signal modulation, and data transmission. Initially, a constant and predefined air pressure is applied to a gauge system. This fixed pressure serves as a reliable baseline for calculating the internal volume of the enclosed space. The core of the system, a PIC32MX550F256L microcontroller, receives this input and uses its internal logic and control units to compute the volume. This microcontroller is programmed to apply a predefined linear mathematical expression that converts this volume data into a corresponding diameter, which represents the main measured output of the system.

Once the diameter is computed, the data is sent to a Neoton IC, which is responsible for modulating the raw digital signals. This IC converts the measurement into a form suitable for both wired and wireless transmission, ensuring that the data remains stable and accurate during communication. For wired communication, the processed signal is transferred through a Serial Input/Output connector to a LAN module based on the W5100 IC, allowing for high-speed and reliable Ethernet-based data delivery to local servers or monitoring stations. In parallel, or as an alternative method, the data is sent to a Bluetooth module (HC-05) along with LoRa communication, enabling long-range wireless transmission for use in remote or mobile scenarios.

At the receiving end, another Neoton IC demodulates the transmitted signal and reconstructs the original diameter data. This recovered information is then displayed on a tablet or display interface, allowing real-time monitoring of parameters such as diameter, voltage and current ratings, and ambient temperature. The system compensates for environmental changes to ensure output consistency, regardless of room conditions such as temperature or humidity. Designed to be low-cost yet highly efficient, the system delivers data with a high accuracy rate of up to 0.1\%, operating at a speed of 9600 bits per second. Its dual-mode communication approach and robust environmental adaptability make it suitable for industrial and technical settings where both precision and versatility are critical.

\subsection{Internal Working of PIC32MX550F256L}
\begin{figure}[ht]
    \centering
    \includegraphics[width=0.6\textwidth]{media/image8.png}
    \caption{Functional Diagram of PIC32MX550F256L microcontroller}
    \label{fig:pic_diagram}
\end{figure}

The PIC32MX550F256L microcontroller is the core processing unit of the ThermoSecure system, managing data acquisition, computation, and communication between various hardware components. Internally, the microcontroller is structured around a high-performance MIPS32 M4K® core capable of operating at high frequencies to ensure timely processing of real-time data. It includes 256 KB Flash memory for program storage and 64 KB SRAM for runtime operations, allowing it to handle complex mathematical calculations and control tasks efficiently.

The internal architecture integrates a variety of built-in peripherals including UART, SPI, I²C, ADC, and USB modules, enabling seamless communication with other devices such as Neoton ICs, LAN modules, and Bluetooth transceivers. The ADC (Analog-to-Digital Converter) is essential for interpreting analog inputs—although in your project the pressure is fixed, this module can still be used for future expansion with analog sensors. The UART interface handles the serial communication between the PIC microcontroller and both the wired (LAN/W5100) and wireless (HC-05 and LoRa) modules.

One of the critical roles of the PIC microcontroller in this project is executing the linear mathematical equation used to calculate the diameter from the given volume based on fixed air pressure. This operation is done in real time using data stored in internal registers and RAM. Once the diameter is computed, the microcontroller encodes the result and sends it to the Neoton IC for modulation.

In addition to data processing, the microcontroller is equipped with onboard USB management that facilitates system updates, debugging, and firmware uploading. It also includes internal voltage and current monitoring mechanisms that contribute to maintaining operational stability and safety under varying environmental conditions.

Overall, the PIC32MX550F256L microcontroller ensures low power consumption, cost-efficiency, and reliable high-speed processing. It serves as the central control hub of the system, enabling the ThermoSecure solution to function effectively in both industrial and mobile settings while maintaining data accuracy, stability, and environmental adaptability.

\subsection{Data Transmission}
\begin{enumerate}
    \item \textbf{Data Collection (Fixed Air Pressure Source):}
    \begin{itemize}
        \item The system uses a predefined fixed air pressure applied to the gauge setup.
        \item This pressure is a known constant and acts as the base input for further calculations.
        \item No external sensor is used to measure this pressure—it is calibrated and controlled manually or by fixed setup values.
        \item The primary goal is to derive volume based on the applied pressure, which is then processed to calculate diameter.
    \end{itemize}
    
    \item \textbf{Data Processing (PIC Microcontroller - PIC32MX550F256L):}
    \begin{itemize}
        \item The PIC microcontroller takes the pressure input and processes it using mathematical expressions based on straight-line equations.
        \item It computes volume, then converts this to diameter using embedded linear calibration logic.
        \item Performs voltage and current evaluation of the circuit, as well as temperature-based correction if required.
        \item Prepares the final data packet containing:
        \begin{itemize}
            \item Computed diameter
            \item Temperature reading (if used)
            \item Voltage \& current rating
            \item Timestamp and status flags (if required)
        \end{itemize}
    \end{itemize}
    
    \item \textbf{Data Transmission via Wired System (LAN - W5100 IC):}
    \begin{itemize}
        \item The PIC communicates with the W5100 Ethernet controller IC through SPI or UART, depending on the configuration.
        \item Data is transmitted through the LAN port to a local server, PC, or monitoring system for real-time tracking or data logging.
        \item Offers high-speed, low-latency communication with minimal packet loss.
        \item Wired transmission ensures reliability in industrial environments.
    \end{itemize}
    
    \item \textbf{Wireless Conversion and Transmission (Neoton IC + LoRa):}
    \begin{itemize}
        \item The data packet is also sent to the Neoton IC, which modulates the signal for wireless transmission.
        \item The Neoton IC handles data encoding and converts it to a format suitable for LoRa or Bluetooth transmission.
        \item This signal is then passed to the LoRa/HC-05 module for wireless broadcast.
        \item Ideal for mobile diagnostics or places where wired LAN access is unavailable.
    \end{itemize}
    
    \item \textbf{Data Reception and Demodulation (Neoton Receiver Module):}
    \begin{itemize}
        \item The Neoton receiver receives the modulated signal over wireless.
        \item It demodulates the data, converting it back to its original digital form.
        \item Data is then relayed to a monitoring device such as a tablet or PC for user visualization or further analysis.
    \end{itemize}
    
    \item \textbf{Data Transmission Rate Summary:}
    \begin{itemize}
        \item Wired (LAN via W5100): Up to 10/100 Mbps (depending on LAN infrastructure).
        \item Wireless (LoRa/Bluetooth via Neoton IC):
        \begin{itemize}
            \item Bluetooth (HC-05): 9600 bps standard (configurable up to 115200 bps)
            \item LoRa: Ranges between 300 bps to 27 kbps depending on settings
        \end{itemize}
        \item Internal data packet: Typically includes 16-24 bytes of diameter, temperature, and rating data, sent at intervals ranging from 100 ms to 1 sec based on system configuration.
    \end{itemize}
\end{enumerate}

\subsection{Algorithm}
\begin{enumerate}
    \item \textbf{Start}
    \item \textbf{Initialize Components}
    \begin{itemize}
        \item PIC Microcontroller (PIC32MX550F256L)
        \item Fixed Air Pressure Source
        \item Serial Input/Output Connector
        \item LAN Module (W5100 IC)
        \item Neoton IC for modulation
        \item Wireless Module (HC-05 or LoRa)
        \item Tablet/PC Display Interface
    \end{itemize}
    
    \item \textbf{Data Collection}
    \begin{itemize}
        \item Fixed air pressure is applied to the gauge system.
        \item The PIC microcontroller calculates volume based on predefined constants and input characteristics.
        \item Using a linear mathematical equation, the volume is converted into diameter.
    \end{itemize}
    
    \item \textbf{Data Processing}
    \begin{itemize}
        \item Voltage and current ratings of the system are measured.
        \item Temperature adjustment logic is applied (if required).
        \item Final data packet is prepared, including:
        \begin{itemize}
            \item Diameter
            \item Voltage \& Current
            \item (Optional) Temperature
            \item Timestamp
        \end{itemize}
    \end{itemize}
    
    \item \textbf{Wired Data Transmission (LAN)}
    \begin{itemize}
        \item The processed data is sent through the LAN port (W5100 IC) to the local monitoring system or server.
        \item Ensures high-speed and stable data communication.
    \end{itemize}
    
    \item \textbf{Wireless Data Conversion and Transmission}
    \begin{itemize}
        \item The same data packet is passed to the Neoton IC, which encodes and modulates the data for wireless transmission.
        \item Data is transmitted via Bluetooth (HC-05) or LoRa based on system configuration.
    \end{itemize}
    
    \item \textbf{Wireless Data Reception}
    \begin{itemize}
        \item The Neoton Receiver demodulates the signal.
        \item Data is displayed on a tablet or PC for visualization.
        \item Used for portable diagnostics or wireless monitoring when LAN is not feasible.
    \end{itemize}
    
    \item \textbf{Repeat the Process}
    \begin{itemize}
        \item Data acquisition and processing loop continues at set intervals or user-defined triggers for real-time monitoring.
    \end{itemize}
    
    \item \textbf{Stop}
    \begin{itemize}
        \item The system halts operations when powered down or manually turned off.
    \end{itemize}
\end{enumerate}

\subsection{Flow Chart}
\begin{figure}[ht]
    \centering
    \includegraphics[width=0.8\textwidth]{media/image9.png}
    \caption{Flow Chart of ThermoSecure Bluetooth-LAN Data System}
    \label{fig:flow_chart}
\end{figure}

% Chapter 4: Results & Discussion
\chapter{Results \& Discussion}
\section{Results}
\subsection{Hardware Results}
\begin{figure}[ht]
    \centering
    \includegraphics[width=0.3\textwidth]{media/image10.jpeg}
    \includegraphics[width=0.3\textwidth]{media/image11.jpeg}
    \includegraphics[width=0.3\textwidth]{media/image12.jpeg}\\
    \includegraphics[width=0.3\textwidth]{media/image13.jpeg}
    \includegraphics[width=0.3\textwidth]{media/image14.jpeg}
    \includegraphics[width=0.3\textwidth]{media/image15.jpeg}\\
    \includegraphics[width=0.9\textwidth]{media/image16.jpeg}\\
    \includegraphics[width=0.3\textwidth]{media/image17.jpeg}
    \includegraphics[width=0.3\textwidth]{media/image18.jpeg}
    \includegraphics[width=0.3\textwidth]{media/image19.jpeg}\\
    \includegraphics[width=0.3\textwidth]{media/image20.jpeg}
    \includegraphics[width=0.3\textwidth]{media/image21.jpeg}
    \includegraphics[width=0.3\textwidth]{media/image22.jpeg}
    \caption{Hardware implementation results}
    \label{fig:hardware_results}
\end{figure}

\subsection{Software Results (Mobile Application)}
\begin{figure}[ht]
    \centering
    \includegraphics[width=0.8\textwidth]{media/image23.png}
    \caption{Mobile application interface}
    \label{fig:mobile_app}
\end{figure}

\subsection{Validation of Results}
\begin{itemize}
    \item \textbf{Range Validation}
    \begin{itemize}
        \item LoRa Mode: Successfully maintained data transmission up to 2.8 km in open areas with negligible data loss.
        \item Bluetooth Mode: Stable connection within 10-15 meters, appropriate for indoor/short-range communication.
        \item Validation: Measured range closely matched the datasheet claims of the modules, proving real-world reliability.
    \end{itemize}
    
    \item \textbf{Data Accuracy and Integrity}
    \begin{itemize}
        \item Approach: Implemented CRC/Checksum to validate incoming and outgoing packets.
        \item Observation: Out of 1000 transmitted packets:
        \begin{itemize}
            \item LoRa mode: 992 packets were correctly received.
            \item Bluetooth mode: 998 packets were correctly received.
            \item LAN mode: All 1000 packets were delivered without errors.
        \end{itemize}
    \end{itemize}
    
    \item \textbf{Error Reduction Mechanism}
    \begin{itemize}
        \item Before Error Handling: Error rate was ~6\% during long-range LoRa transmission.
        \item After Implementation of Retransmission Logic \& CRC: Error rate reduced to less than 1\%.
        \item Validation: Confirmed by logging all packet transmissions and analyzing logs.
    \end{itemize}
    
    \item \textbf{Power Consumption Validation}
    \begin{itemize}
        \item Measured power consumption during LoRa transmission using a multimeter and data logger.
        \item Observation: LoRa module consumed minimal power during sleep mode (<1µA) and average 120mA during transmission.
    \end{itemize}
\end{itemize}

\subsection{Sample Calculations}
\subsubsection{Volume-to-Diameter Conversion}
\begin{itemize}
    \item Type: Mathematical / Linear Equation
    \item Used For: Calculating diameter from known volume (via fixed pressure)
    \item Formula: Diameter = m × Volume + c
    \item Example: m = 0.5, c = 2, Volume = 20 cm³
    \item Diameter = 0.5 × 20 + 2 = 12 mm
\end{itemize}

\subsubsection{Linear Calibration Equation}
\begin{itemize}
    \item Type: Calibration-based linearity
    \item Used For: Ensuring consistent output under pressure conditions
    \item Formula: Y = mX + C
    \item Example: X = 1.2 V, m = 10, C = 0
    \item Output Y = 12 mm
\end{itemize}

\subsubsection{Error Rate Calculation}
\begin{itemize}
    \item Type: Communication performance evaluation
    \item Formula: Error Rate = (Erroneous packets / Total packets) × 100\%
    \item Example: 8 errors out of 1000 packets
    \item Error Rate = (8 / 1000) × 100 = 0.8\%
\end{itemize}

\subsubsection{Power Consumption Analysis}
\begin{itemize}
    \item Type: Electrical measurement
    \item Formula: P = V × I
    \item Example: V = 3.3V, I = 120mA
    \item Power = 3.3 × 0.12 = 0.396 W
\end{itemize}

\subsubsection{Data Transmission Rate}
\begin{itemize}
    \item Type: Communication Speed
    \item LAN: 10/100 Mbps
    \item Bluetooth HC-05: 9600 bps
    \item LoRa: 300 bps to 27 kbps
    \item Example (Bluetooth, 24-byte packet): Time = (24 × 8) / 9600 = 0.02 sec
\end{itemize}

\subsubsection{Accuracy Validation}
\begin{itemize}
    \item Type: Data accuracy
    \item Given Accuracy: ±0.1\%
    \item Example: Expected = 12.0 mm, Measured = 11.988 mm
    \item Error = (|12.0 - 11.988| / 12.0) × 100 = 0.1\%
\end{itemize}

\subsubsection{Transmission Success Rate}
\begin{itemize}
    \item Type: Reliability Metric
    \item Formula: Success Rate = (Valid packets / Total packets) × 100\%
    \item Example: 1000 sent and received
    \item Success Rate = (1000 / 1000) × 100 = 100\%
\end{itemize}

\section{Discussion}
% Discussion content goes here

% Chapter 5: Conclusions
\chapter{Conclusions}
\section{Conclusions}
The ThermoSecure Bluetooth-LAN Data System successfully integrates Bluetooth and LAN communication to enable fast, secure, and error-free data transmission. By leveraging the PIC32MX550F256L microcontroller, W5100 Ethernet IC, and HC-05 or LoRa Bluetooth module, the system ensures efficient data transfer with minimal latency and reduced packet loss. The project demonstrates significant improvements in data security, transmission speed, and reliability, making it highly suitable for applications in industrial automation, healthcare, smart homes, and military communication. The implementation of error detection and correction mechanisms enhances system stability, ensuring accurate data exchange between Bluetooth-enabled devices and LAN-based networks. Overall, ThermoSecure provides an innovative solution for bridging wireless and wired networks, offering a robust and scalable framework for real-time data transmission in various critical applications. Future enhancements can include encryption techniques for better security, AI-based error prediction, and integration with cloud storage for advanced data management.

\section{Future Scope}
\begin{itemize}
    \item \textbf{Cloud Integration \& IoT Expansion:} Extend the system to sync data with cloud platforms like Firebase, AWS, or Azure for remote monitoring, analytics, and real-time alerts.
    \item \textbf{Cross-Platform Mobile App:} Develop a mobile application (Android/iOS) to control and monitor the system, improving accessibility and user experience.
    \item \textbf{Stronger Encryption \& Cybersecurity:} Implement AES or RSA encryption and secure communication protocols (e.g., HTTPS, TLS) to enhance data security across networks.
    \item \textbf{AI-Based Error Prediction:} Integrate machine learning to predict and automatically correct transmission errors based on historical patterns.
    \item \textbf{BLE or Wi-Fi 6 Upgrade:} Replace HC-05 with BLE or Wi-Fi 6 modules for better data rate, power efficiency, and long-range communication.
    \item \textbf{Industrial Adaptation:} Deploy the system in industrial environments for secure sensor-to-server communication or machine-to-machine (M2M) networking.
    \item \textbf{Data Visualization Dashboard:} Build a web-based dashboard for visualizing system logs, data packets, transmission speed, and error statistics in real-time.
    \item \textbf{Multi-Device Support:} Scale the system to handle multiple Bluetooth inputs and simultaneously route them over a single LAN network.
    \item \textbf{Power-Efficient Design:} Implement sleep modes, power gating, or use of low-power microcontrollers to optimize energy usage for battery-based deployments.
    \item \textbf{Custom Protocol Translation:} Introduce protocol translation features, allowing the system to convert Bluetooth signals into HTTP, MQTT, or Modbus for industrial standards.
\end{itemize}

% References
\begin{thebibliography}{9}
\bibitem{mohan2018} 
Mohan, N., Krishnamurthy, S., \& Gupta, A. (2018). Adaptive Modulation Techniques for RF Communication under Varying Thermal Conditions. IEEE Transactions on Wireless Communications, 17(9), 6184-6194.

\bibitem{qian2019} 
Qian, L., Zhang, Y., \& Zhou, X. (2019). Multi-format Data Transmission Architecture for Industrial IoT Applications. Sensors, 19(5), 1058.

\bibitem{suzuki2020} 
Suzuki, H., Tanaka, T., \& Kato, Y. (2020). Temperature-compensated Frequency Synthesizer for Stable Wireless Communication. Journal of Electrical Engineering \& Technology, 15(3), 1234-1242.

\bibitem{barrenetxea2008} 
Barrenetxea, G., Ingelrest, F., Schaefer, G., \& Vetterli, M. (2008). Wireless Sensor Networks for Environmental Monitoring: The SensorScope Experience. Proceedings of the IEEE International Conference on Local Computer Networks, 10, 20-28.

\bibitem{berrou1993} 
Berrou, C., Glavieux, A., \& Thitimajshima, P. (1993). Near Shannon Limit Error-Correcting Coding and Decoding: Turbo Codes. Proceedings of IEEE ICC, 2, 1064-1070.

\bibitem{patel2021} 
Patel, R., Thomas, A., \& Reddy, V. (2021). Thermal-Aware RF Network Design for Industrial Automation. International Journal of Wireless Information Networks, 28, 119-130.

\bibitem{karthikeyan2019} 
Karthikeyan, B., \& Rao, R. (2019). Real-Time Temperature Monitoring Using PIC32 Microcontroller in Edge Computing Environments. International Journal of Embedded Systems, 11(4), 279-287.

\bibitem{ieee2017} 
IEEE IoT Standards Committee. (2017). Low-Power and Lightweight Protocols for IoT Devices: A Review. IEEE Communications Standards Magazine, 1(3), 92-97.

\bibitem{dey2022} 
Dey, S., Paul, S., \& Bandyopadhyay, S. (2022). Impact of Temperature Variation on Patient Monitoring Devices in Clinical Environments. Journal of Biomedical Engineering and Informatics, 8(2), 45-54.

\bibitem{nasa2016} 
NASA Jet Propulsion Laboratory. (2016). Multi-format Telemetry Systems for Deep Space Communication. Technical Report, Pasadena, California.

\bibitem{gonzalez2021} 
Gonzalez, A., Mendes, J., \& Pinto, H. (2021). Temperature-Aware Edge Computing with Lightweight Communication Protocols. Sensors, 21(12), 3983.

\bibitem{gupta2020} 
Gupta, N., Sharma, D., \& Sinha, M. (2020). Modular Sensor Platforms for Industrial IoT: Design and Deployment Case Studies. International Journal of Advanced Computer Science and Applications, 11(6), 189-195.
\end{thebibliography}

\end{document}
